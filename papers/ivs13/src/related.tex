The problem of sensor calibration has been a recurring one in the history of
robotics and computer vision. Thereby, it has been addressed using a
variety of sensor setups and algorithms. A calibration process may involve
recovering \emph{intrinsic}, e.g. focal length for a camera, and
\emph{extrinsic} parameters, i.e. the rigid transformation between the sensor's
coordinate system and a reference coordinate system. For the former, one could
devise a naive approach where the parameters are accurately determined during
the manufacturing process. Similarly, the transformation could be retrieved by
means of some measuring instrument. However, the disadvantages of such purely
engineered methods are manifold. Apart from their impracticality, it can be
nearly impossible to reach a satisfying accuracy and thus hinder the proper use
of the sensor in a robotic system. Furthermore, external factors such as
temperature variations or mechanical shocks may seriously bias a factory
calibration. Therefore, much efforts have been dedicated over the years to
develop algorithms for calibration of systems in the field.

The use of a known calibration pattern such as a checkerboard coupled with
nonlinear regression has become the most popular method in computer vision
during the last decade. It has been deployed both for intrinsic camera
calibration~\cite{sturm99plane} and extrinsic calibration
between heterogeneous sensors~\cite{zhang04extrinsic}. While being relatively
efficient, this procedure still requires expert knowledge to reach a good level
of accuracy. It can also be quite inconvenient on a mobile platform requiring
frequent recalibration.

In an effort to automate the process in the context of mobile robotics, several
authors have included the calibration problem in a state-space estimation
framework, either with filtering~\cite{martinelli06automatic} or
smoothing~\cite{kuemmerle11simultaneous} techniques. Filtering techniques based
on the Kalman filter are appealing due to their inherently online nature.
However, in case of nonlinear systems, smoothing techniques based on iterative
optimization are usually superior in terms of accuracy. While building on this
latter method, our approach attempts to reduce its computational load and copes
with degenerate cases.

More recently, Brookshire and Teller~\cite{brookshire12extrinsic} have carried
out a formal observability analysis in order to identify degenerate paths of
the calibration run. In contrast to the method presented in this paper, their
approach still expects that the robot travels along non-degenerate paths.

A last class of methods relies on an energy function to be minimized. For
instance, Levinson and Thrun~\cite{levinson10unsupervised} have defined an
energy function based on surfaces and Sheehan \emph{et al.}~\cite{sheehan12self}
on an information theoretic quantity measuring point cloud quality.

To the best of the authors' knowledge, little research has been devoted to
efficiently deal with degenerate cases frequently occurring during calibration.
The majority of the authors indeed assumes that their optimization routine runs
on well-behaved data. As demonstrated in the next sections, this can be a
critical point in real-world scenarios. In contrast, our approach does not
require any of these assumptions and is suitable for online, autonomous, and
long-term operations on various platforms and sensors.
