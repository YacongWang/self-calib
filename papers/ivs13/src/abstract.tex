We present a generic algorithm for self calibration of robotic systems
that utilizes two key innovations. First, it uses information
theoretic measures to automatically identify and store novel
measurement sequences. This keeps the computation tractable by
discarding redundant information and allows the system to build a
sparse but complete calibration dataset from data collected at
different times. Second, as the full observability of the calibration
parameters may not be guaranteed for an arbitrary measurement
sequence, the algorithm detects and locks unobservable directions in
parameter space using a truncated QR decomposition of the Gauss-Newton
system. The result is an algorithm that listens to an incoming
sensor stream, builds a minimal set of data for estimating the
calibration parameters, and updates parameters as they become
observable, leaving the others locked at their initial guess.

Through an extensive set of simulated and real-world experiments, we
demonstrate that our method outperforms state-of-the-art algorithms in
terms of stability, accuracy, and computational efficiency.
