% In this paper, we address the problem of sensors calibration for mobile robots.
% In particular, we are aiming towards an automated system ready to be deployed
% for long-term and online operation in the hands of non-experts. To this end, we
% augment the classical SLAM formulation with calibration parameters and derive
% an MAP estimator that is computed using a Gauss-Newton algorithm. Degenerate
% dimensions in sensor data are identified by means of QR decomposition and
% automatically discarded from the optimization via a truncated method, such that
% only observable parts are exploited. Finally, incoming sensor data is processed
% in batch and selected using an information theoretic measure. Through an
% extensive set of simulated and real-world experiments, we demonstrate that our
% method outperforms state-of-the-art algorithms in terms of applicability,
% accuracy, and speed.

%Every robotic system has some set of calibration parameters that must
%be accurately known for safe operation and accurate state estimation,
%planning, and control. Despite best efforts, some
%parameters will change over the robot's lifetime due to normal wear
%and tear. 

We present a generic algorithm for self calibration of robotic systems
that utilizes two key innovations. First, it uses information
theoretic measures to automatically identify and store novel
measurement sequences. This keeps the computation tractable by
discarding redundant information and allows the system to build a
sparse but complete calibration dataset from data collected at
different times. Second, as the full observability of the calibration
parameters may not be guaranteed for an arbitrary measurement
sequence, the algorithm detects and locks unobservable directions in
parameter space using a truncated QR decomposition of the Gauss-Newton
system. The result is an algorithm that listens to an incoming
sensor stream, builds a minimal set of data for estimating the
calibration parameters, and updates parameters as they become
observable, leaving the others locked at their initial guess.

Through an extensive set of simulated and real-world experiments, we
demonstrate that our method outperforms state-of-the-art algorithms in
terms of stability, accuracy, and computational efficiency.
