The problem of sensor calibration has been a recurring one in the history of
robotics and computer vision. Thereby, it has been addressed using a
variety of sensor setups and algorithms. A calibration process may involve
recovering \emph{intrisic}, e.g. focal length for a camera, and \emph{extrinsic}
parameters, i.e. the rigid transformation between the sensor's coordinate system
and a reference coordinate system. For the former, one could devise a naive
approach where the parameters are accurately determined during the manufacturing
process. In the same vein, the transformation could be retrieved by means of
some measuring instrument. However, the disadvantages of such purely
engineered methods are manifolds. Apart from their impracticality, it can be
nearly impossible to reach a satisfying accuracy and thus hinder the proper use
of the sensor in a robotic system. Furthermore, external factors such as
temperature variation or mechanical shocks may bias a factory calibration.

List of papers to cite: \cite{davis11algorithm},
\cite{brookshire12extrinsic}, \cite{aster11parameter}, \cite{brookshire11automatic},
\cite{chan94lowrank}, \cite{finsterle11truncated}, \cite{golub96matrix},
\cite{hansen87truncated}, \cite{hong92rank}, \cite{jauffret07observability},
\cite{kitagawa01regularization}, \cite{levinson10unsupervised},
\cite{mackay05information}, \cite{maddern12lost}, \cite{martinelli06automatic},
\cite{sheehan12self}, \cite{sheehan10automatic}, \cite{zhang04extrinsic},
\cite{gao10online}, \cite{hol10modeling}
