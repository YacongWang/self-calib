The problem of sensor calibration has been a recurring one in the history of
robotics and computer vision. Thereby, it has been addressed using a
variety of sensor setups and algorithms. A calibration process may involve
recovering \emph{intrisic}, e.g. focal length for a camera, and \emph{extrinsic}
parameters, i.e. the rigid transformation between the sensor's coordinate system
and a reference coordinate system. For the former, one could devise a naive
approach where the parameters are accurately determined during the manufacturing
process. In the same vein, the transformation could be retrieved by means of
some measuring instrument. However, the disadvantages of such purely
engineered methods are manifolds. Apart from their impracticality, it can be
nearly impossible to reach a satisfying accuracy and thus hinder the proper use
of the sensor in a robotic system. Furthermore, external factors such as
temperature variations or mechanical shocks may seriously bias a factory
calibration. Therefore, much efforts have been dedicated over the years to
develop algorithms easing custom calibration on the field.

The use of a known calibration pattern such as a checkerboard coupled with
nonlinear regression has become the most popular method in computer vision
during the last decade. It has been deployed both for intrinsic camera
calibration~\cite{sturm99plane} and extrinsic calibration
between heterogeneous sensors~\cite{zhang04extrinsic}. While being relatively
efficient, this procedure still requires expert knowledge to reach a good level
of accuracy. It can also be quite inconvenient on a mobile platform requiring
frequent recalibration.

In the context of mobile robotics, several authors have included the calibration
problem in a state-space estimation framework,
either as filtering~\cite{martinelli06automatic}, or
smoothing~\cite{kuemmerle11simultaneous}.

Last class of approach: \cite{brookshire12extrinsic}, \cite{brookshire11automatic}, \cite{maddern12lost},
\cite{sheehan12self}, \cite{sheehan10automatic}, \cite{levinson10unsupervised}, \cite{gao10online},
\cite{hol10modeling},

Theoretical citations: \cite{davis11algorithm}, \cite{aster11parameter},
\cite{chan94lowrank}, \cite{finsterle11truncated}, \cite{golub96matrix},
\cite{hansen87truncated}, \cite{hong92rank}, \cite{jauffret07observability},
\cite{kitagawa01regularization},
\cite{mackay05information}, 
 \cite{kaess09covariance}
