\subsection{Problem Formulation\label{subsec:prob}}

In the following, we outline our problem in a probabilistic manner, closely
following the discrete-time Simultaneous Localization and Mapping (SLAM)
formulation~\cite{durrantwhyte06simultaneous}. For the sake of clarity, we
consider here a robot with a single sensor observing a known number of landmarks
at each timestep. Furthermore, we assume the correspondences between sensor's
measurements and landmarks are known. The relaxation of these assumptions goes
beyond the scope of this paper.

Let
$\mathcal{X}=\{\mathbf{x}_{0:K}\}$ be a set of \emph{latent} random variables
(LRV) representing robot states up to timestep $K$,
$\mathcal{U}=\{\mathbf{u}_{1:K}\}$ a set of \emph{observable} random variables
(ORV) representing measured control inputs, $\mathcal{L}=\{\mathbf{l}_{1:N}\}$ a
set of LRV representing $N$ landmarks' positions,
$\mathcal{Z}=\{\mathbf{z}_{1_{1:N}:K_{1:N}}\}$ a set of ORV representing $KxN$
landmarks' measurements, and $\boldsymbol{\Theta}$ a LRV representing the
calibration parameters of the robot's sensor. The goal of the calibration
procedure is to compute the posterior marginal distribution of
$\boldsymbol{\Theta}$ given all the measurements up to timestep $K$

\begin{equation}\label{eqn:post_calib}
  \begin{aligned}
  p(\boldsymbol{\Theta}\mid\mathcal{U},\mathcal{Z}) &=
    \int_{\mathcal{X}, \mathcal{L}}p(\boldsymbol{\Theta}, \mathcal{X},
    \mathcal{L} \mid\mathcal{U},\mathcal{Z}).
  \end{aligned}
\end{equation}

The full joint posterior on the left-hand side of \eqref{eqn:post_calib} may
further be factorized into

\begin{equation}\label{eqn:post_joint_factorized}
  \begin{aligned}
  p(\boldsymbol{\Theta}, \mathcal{X},
    \mathcal{L} \mid\mathcal{U},\mathcal{Z}) &\propto\\
    p(\boldsymbol{\Theta}, \mathbf{x}_0,\mathcal{L})
    \prod_{k=1}^K p(\mathbf{x}_k\mid\mathbf{x}_{k - 1},\mathbf{u}_k)
    \prod_{k=1}^K\prod_{i=1}^N p(\mathbf{z}_{k_i}\mid\mathbf{x}_k,
    \mathbf{l}_i,\boldsymbol{\Theta}).
  \end{aligned}
\end{equation}

We may approximate \eqref{eqn:post_joint_factorized} with a normal distribution
whose moments $\boldsymbol{\mu}_{\boldsymbol{\Theta}\mathcal{X}\mathcal{L}}$ and
$\boldsymbol{\Sigma}_{\boldsymbol{\Theta}\mathcal{X}\mathcal{L}}$ have to be
estimated. To this end, we first derive a Maximum a Posteriori (MAP) estimator
for the mean

\begin{equation}\label{eqn:map_estimator}
  \begin{aligned}
  \hat{\boldsymbol{\mu}}_{\boldsymbol{\Theta}\mathcal{X}\mathcal{L}} &=
    \argmax_{\boldsymbol{\Theta},\mathcal{X},\mathcal{L}}
    p(\boldsymbol{\Theta}, \mathcal{X},\mathcal{L} \mid\mathcal{U},\mathcal{Z})
    \\
    &= \argmin_{\boldsymbol{\Theta},\mathcal{X},\mathcal{L}}-\log
    p(\boldsymbol{\Theta}, \mathcal{X},\mathcal{L} \mid\mathcal{U},\mathcal{Z}).
  \end{aligned}
\end{equation}

We further refine our problem by defining a \emph{motion} and \emph{observation}
model

\begin{equation}\label{eqn:process_model}
  \begin{aligned}
  \mathbf{x}_k &= \mathbf{h}(\mathbf{x}_{k-1}, \mathbf{u}_k, \mathbf{w}_k)\\
  \mathbf{z}_{k_i} &= \mathbf{g}(\mathbf{x}_{k}, \mathbf{l}_i, \mathbf{n}_k),
  \end{aligned}
\end{equation}

where

\begin{equation}\label{eqn:noise_model}
  \begin{aligned}
  \mathbf{w}_k \sim \mathcal{N}(\mathbf{0},\mathbf{Q}_k)\\
  \mathbf{n}_k \sim \mathcal{N}(\mathbf{0},\mathbf{R}_k)
  \end{aligned}
\end{equation}

are normally distributed process and observation noise variables,
$\mathbf{Q}_k$ and $\mathbf{R}_k$ being known covariance matrices. The functions
$\mathbf{h}(\cdot)$ and $\mathbf{g}(\cdot)$ might be non-linear in their
parameters.

\subsection{Least Squares Solution}
In case of linear motion and observation models, there exists a closed-form
solution to \eqref{eqn:map_estimator} based on the \emph{least squares} method
due to the normally distributed noise variables. In the other case, one can
resort to non-linear least squares methods that iteratively solve a linearized
version of the problem. In the following, we employ the \emph{Gauss-Newton}
algorithm for this purpose.



\subsection{Truncated QR Solution}

\subsection{Selecting Informative Measurements}

